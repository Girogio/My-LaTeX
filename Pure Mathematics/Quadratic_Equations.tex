\documentclass{standalone}
\begin{document}
	\chapter{Quadratic Equations}
	\section{Definition}
	\quad A quadratic equation is of the form $ax^2+bx+c=0$ where $a,b,c \in \mathbb{R},\quad a\neq0$. These can be solved algebraically using one of the following methods:
	\begin{itemize}		
		\item{Fractions}
		\item{Completing the square}
		\item{The quadratic formula\footnote[3]{\qe}}
	\end{itemize}
	\section{Nature of roots of the Quadratic Equation}
	\quad Any quadratic equation has in general two roots, namely \qe.
	The quantity $b^2-4ac$ determines the nature of these roots.
	\begin{itemize}
		\item{$b^2-4ac > 0\colon$ Equation holds two real and distinct roots.}
		\item{$b^2-4ac = 0\colon$ Equation holds two equal\footnote[4]{It is implied that they are real.} roots.}
		\item{$b^2-4ac > 0\colon$ Equation holds two complex roots.}
	\end{itemize}
	\quad Thus, the quantity $b^2-4ac$ discriminates among the type of roots that a quadratic equation may have. Therefore it is called the discriminant.
	\begin{example}
		Determine, without solving the nature of the following function.
	\end{example}
	\begin{alignat*}{2}
		& Let        & f(x) =                 & 2x^2+3x-17                              \\
		& \implies   & b^2-4ac                & =3^2-4(2)(-17)                          \\
		&            &                        & = 145                                   \\
		&            &                        & > 0                                     \\
		& \therefore & \text{Roots of $f(x)$} & \in \mathbb{R} \text{ and are distinct} 
	\end{alignat*}
	\hrulefill
	\begin{example}
		Determine the value of $p$ if $ px^2-10x+1 = 0 $ has two equal roots
	\end{example}
	
	Given that the equation has two equal roots:
	
	\begin{alignat*}{2}
		&          & b^2-4ac                        & = 0 \\
		& \implies & 100-4p                         & = 0 \\
		&          & \boxed{\therefore \quad p = 5} &     
	\end{alignat*}
	
	\newpage
	\section{Roots and Coefficients of a Quadratic Equation}
	\subsection{Proof}
	\emph{Consider a general quadratic equation:}
	\begin{alignat*}{2}
		&            & ax^2+bx+c                            & = 0 \tag{1..}                              \\
		& \implies   & x^2 +\frac{bx}{a} + \frac{c}{a}      & = 0                                        \\
		\intertext{Let $\alpha$  and $\beta$  be the roots:}
		& \implies   & (x-\alpha)(x-\beta)                  & = 0                                        \\
		& \implies   & x^2 -\beta x -\alpha x + \alpha\beta & = 0                                        \\
		& \implies   & x^2-(\alpha + \beta)x + \alpha\beta  & = 0\tag{2..}                               \\
		\intertext{Since (1..) and (2..) are identical:}
		& \implies   & x^2 +\frac{bx}{a} + \frac{c}{a}      & \equiv x^2-(\alpha + \beta)x + \alpha\beta \\
		& \therefore & \alpha + \beta                       & = \frac{-b}{a}                             \\ 
		&            & \alpha\beta                          & = \frac{c}{a}                              
	\end{alignat*}
	\hrulefill
	
	\begin{example}
		Write down the quadratic equation whose roots have a sum of 7 \& a product of 5.
	\end{example}
	\begin{alignat*}{2}
		&   &        & x^{2} - (\alpha + \beta) + (\alpha\beta) \\
		&   & =\quad & x^2 - 7x + 7                             
	\end{alignat*}
	\hrulefill
	\begin{example}
		The roots of the equation $2x^2 + 5x -1$ are $\alpha$ and $\beta$. Find the equation whose roots are $\frac{1}{\alpha} $ \& $\frac{1}{\beta}$
	\end{example}
	\begin{alignat*}{2}
		&   & \alpha + \beta & = \frac{-5}{2} \\
		&   & \alpha\beta    & = \frac{-1}{2} 
	\end{alignat*}
	\newpage
	\begin{multicols}{2}
		\textit{Sum of roots:}
		\begin{alignat*}{2}
			&   &   & \frac{1}{\alpha} + \frac{1}{\beta} \\
			&   & = & \frac{\alpha+\beta}{\alpha\beta}   \\
			&   & = & \frac{-5}{2} \div \frac{-1}{2}     \\
			&   & = & 5                                  
		\end{alignat*}
		\textit{Product of roots:}
		
		\begin{alignat*}{2}
			\\&&&\frac{1}{\alpha\beta}\\
			&   & = & \frac{-2}{1} \\
			&   & = & -2           
		\end{alignat*}
	\end{multicols}
	\[ 
	\therefore \quad f(x) = x^2-5x-2
	\]
	\newpage
	
	\end{document}